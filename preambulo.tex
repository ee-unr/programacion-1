% VARIAS CONFIGURACIONESS  ----------------------------------------------------

\usepackage{fontspec}

% Colores
\usepackage{colortbl, xcolor}
\definecolor{oiB}{HTML}{000000}            % COL["blue","full"]
\definecolor{oiLB}{HTML}{e0e0e0}           % lighter version of oiB

\definecolor{oiY}{HTML}{000000}            % COL["yellow","full"]
\definecolor{oiLY}{HTML}{e0e0e0}           % lighter version of oiY

\definecolor{oiR}{HTML}{000000}            % COL["red","full"]
\definecolor{oiLR}{HTML}{e0e0e0}           % lighter version of oiR

\definecolor{oiGray}{HTML}{808080}         % COL["gray","full"]
\definecolor{oiLGray}{HTML}{f8f8f8}        % lighter version of oiR

\definecolor{oiGB}{rgb}{0.5,0.5,.5}        % from OS4 - for footnotes

% Márgenes 
\usepackage[top=3cm,bottom=3cm,left=1.5cm,right=3cm]{geometry}

% TOC
% que no tenga el nro de parte, porque eso se lo pongo yo en el título
% Deshabilitar la numeración en la tabla de contenidos
% \renewcommand{\thepart}{\relax}
% ESO NO, LO SACA DE TODOS LADOS

% Redefinir el comando \part para evitar la carátula 
\usepackage[explicit]{titlesec}
% esto hace que el texto de la parte este a continuacion del titulo
\titleclass{\part}{top}
% esto modifica el formato del titulo
\titleformat{\part}[display]
{\color{oiR}\titlerule[5pt]\vspace{3pt}\color{oiLR}\titlerule[2pt]\color{oiB}\normalfont\Huge\bfseries\scshape}
{}{0em}{#1 \\ \noindent \vspace{3pt}\color{oiR}\titlerule[5pt]}
\titlespacing*{\part}{0pt}{*0}{2em}


% \makeatletter
% \renewcommand\part{\cleardoublepage
%   \thispagestyle{plain}% Usar un estilo de página sencillo
%   \if@twocolumn
%     \onecolumn
%     \@tempswatrue
%   \else
%     \@tempswafalse
%   \fi
%   \secdef\@part\@spart}
% \makeatother

% Encabezado y pie de página --------------------------------------------------

% Paquete para personalizar encabezados y pies de página
\usepackage{fancyhdr}
\usepackage{lipsum} % Para texto de ejemplo

% Configuración de fancyhdr
\pagestyle{fancy}

% Limpiar estilos predeterminados
\fancyhf{}

% Configuración del contenido
\fancyhead[LO, RE]{Unidad \thepart}
\fancyhead[RO, LE]{\nouppercase{\leftmark}}
\fancyfoot[LO, RE]{Programación 1}
\fancyfoot[RO, LE]{\thepage}

\renewcommand{\footrulewidth}{0.4pt}
\renewcommand{\sectionmark}[1]{\markright{#1}}

% para que la primera pagina no tenga el header pero si el footer como el resto
\fancypagestyle{primerapagina}{
  \fancyhead[LO, RE]{Unidad \thepart}
  \fancyhead[RO, LE]{\nouppercase{\leftmark}}
  \fancyfoot[LO, RE]{Programación 1}
  \fancyfoot[RO, LE]{\thepage}
	% \fancyhead{}
	% \fancyfoot{}
	% \renewcommand{\headrulewidth}{0pt}
}

% Para que diga "Unidad" antes del nro y nombre de la unidad
% \usepackage{titlesec}
% \titleformat{\chapter}{\normalfont\LARGE\bfseries}{Unidad \thechapter.}{0em}{}

% Para que haya encabezado en la primera hoja de cada cap tmb
% \usepackage{etoolbox}
% \patchcmd{\chapter}{\thispagestyle{plain}}{\thispagestyle{fancy}}{}{}

% Pero lo anterior agrega encabezad en la tabla de contenidos, esto lo evita
% \AtBeginDocument{%
% 	\addtocontents{toc}{\protect\thispagestyle{empty}}
% }

% fuente ESETO NO FUNCIONO Y NO SEGUI PROBANDO ---------------------------------
% \usepackage[T1]{fontenc}
% \usepackage{carlito}
% \renewcommand*\oldstylenums[1]{\carlitoOsF #1}
% \renewcommand{\familydefault}{\sfdefault}

% Formato de títulos ----------------------------------------------------------

% Basado en IMS https://github.com/OpenIntroStat/ims/blob/main/latex/ims-style.tex
\titleformat{\chapter}[display]
{\color{oiB}\normalfont\Huge\bfseries}
{\color{oiB}Chapter \thechapter}{1em}{#1}

\titleformat{\section}
{\color{oiB}\normalfont\Large\bfseries}
{\color{oiB}\thesection}{1em}{#1}

\titleformat{\subsection}
{\color{oiB}\normalfont\large\bfseries}
{\color{oiB}\thesubsection}{1em}{#1}

\titleformat{\subsubsection}
{\color{oiB}\normalfont\normalsize\bfseries}
{\color{oiB}\thesubsubsection}{1em}{#1}
